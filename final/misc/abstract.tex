\begin{cabstract}
  \renewcommand{\chapterlabel}{摘\hspace{2em}要}

贵金属交易一般指投资者在对贵金属市场看好的情形下,低价买入高价卖出,从而赚取差价
的过程。同时也可以是投资人在经济低迷的情况下所采取的一种避险手段,以实现资产的保
值增值。贵金属由于其较高的价值,近年来成为了人们投资以及保值的重要工具。与此同
时,随着金融创新的发展以及金融危机的频发,掌握贵金属的市场波动和投资风险在对金融
交易市场的研究中显得尤为重要。由于不同贵金属之间具有一定相关性,单个市场的价格波
动可能会造成其他贵金属的价格变化,因此研究贵金属市场间的相依关系,对投资者规避投
资风险以及管理部门维护市场稳定有着重要意义。在贵金属市场交易中,黄金和白银一直占
有极大比例。黄金和白银在历史上都是硬通货的代表。一直以来,黄金的货币属性较白银强
势,避险、保值和抗通胀特性突出,白银的货币属性虽弱于黄金,但其具有很强的工业属
性,因此,黄金和白银的交易可看作贵金属市场变动的“晴雨表”,这两者价格和交易量可实
时反映贵金属市场变化以及宏观股市环境的动态。

随着技术的发展,高频数据的可得性越来越高,本文选取了在贵金属现货市场中交易量占比
在95\%以上的黄金T+D和白银T+D代表贵金属现货市场,收集黄金T+D和白银T+D的5分钟高频
交易数据,利用5分钟高频收盘价数据计算不同的已实现测度,即波动估计量。首先分别对
黄金和白银单个波动特征进行研究分析,基于不同的已实现测度和收益率数据进行描述性分
析和基本检验,发现黄金和白银的对数收益率序列均具有尖峰厚尾和波动聚集特征。然后分
别对黄金和白银基于不同分布假设构建一元Realized Garch模型,对不同模型参数估计结果
进行样本内极大似然函数值比较和样本外Kupiec失败率检验法VaR预测效果比较,选择最优
模型。研究发现对黄金T+D和白银T+D对数收益率序列,在广义双曲线分布假设下基于medRV
作为已实现测度方式的 Realised GARCH(1,1) 拟合效果最好。通过基于最优模型对黄金和
白银波动特征研究发现,黄金和白银收益率均存在杠杆效应,负向收益率冲击对波动率的影
响要比正向收益率对波动率的影响更大。同时结合多元的Realized Wishart-Garch模型研究
黄金和白银的动态相关关系,实证发现黄金与白银具有较强的正向相关关系,且这种关系具
有持续性,因此,投资者应当注意防范贵金属价格同向变动可能会造成的损失。


  
  \bigbreak

  {\bfseries 关键词}:Realized Garch;贵金属;波动特征;动态相关性
   
\blankpage
\end{cabstract}




\begin{eabstract}

Precious metal trading generally refers to investors who are optimistic about
the precious metal market, buying at low prices and selling at high prices to
make a difference the process of. At the same time, it can also be a hedging
method taken by investors in an economic downturn to achieve asset protection.
Value appreciation. Because of its high value, precious metals have become an
important tool for people to invest and maintain value in recent years. And with
gold The development of financial innovation and the frequent occurrence of
financial crises, grasp the market fluctuations and investment risks of precious
metals in the financial transaction market It has an important position in
research. Due to the correlation between different precious metals, price
fluctuations in a single market may cause Changes in the price of other precious
metals. Therefore, it is necessary to study the interdependence between the
precious metal markets to avoid investment risks and It is of great significance
for the management department to maintain market stability. In precious metals
market transactions, gold and silver have always occupied a huge proportion
example. Both gold and silver have historically represented hard currencies. The
currency attribute of gold has always been stronger than that of silver. The
characteristics of insurance, value preservation and anti-inflation are
outstanding. Although silver’s currency attributes are weaker than gold, it has
strong industrial attributes. This gold and silver transaction can be regarded
as a "barometer" of changes in the precious metal market, and its price and
transaction volume can reflect the precious metal market in real time. Market
changes and the dynamics of the macro stock market environment.

With the development of technology, the availability of high-frequency data is
increasing. This article selects the proportion of trading volume in the
precious metal spot market Au(T+D) and Ag(T+D) above 95\% represent the precious
metal spot market. Collect Au(T+D) and Ag(T+D) for 5-minute high-frequency
trading Easydata uses 5-minute high-frequency closing price data to calculate
different realized measures, namely volatility estimates. First of all Research
and analyze individual fluctuation characteristics of gold and silver, and
perform descriptive analysis based on different realized measures and yield data
And basic inspection, it is found that the logarithmic return series of gold and
silver have the characteristics of sharp peaks and thick tails and volatility
clustering. Then separately For gold and silver based on different distribution
assumptions, a one-element Realized Garch model is constructed, and the
estimation results of different model parameters are analyzed. Compare the value
of the maximum likelihood function within the sample and compare the VaR
prediction effect of the Kupiec failure rate test method outside the sample, and
select the best model. The study found that the logarithmic return series of
Au(T+D) and Ag(T+D) are based on medRV under the assumption of generalized
hyperbolic distribution. Realised GARCH(1,1), which is an implemented
measurement method, has the best fitting effect. Based on the optimal model for
gold and Research on the characteristics of silver volatility found that both
gold and silver yields have a leverage effect, and the impact of negative yield
shocks on volatility The impact is greater than the positive rate of return on
volatility. Then combined with the multiple Realized Wishart-Garch model
research The dynamic correlation between gold and silver, empirical findings
have found that gold and silver have a strong positive correlation, and this
relationship has Sustainability, investors should pay attention to prevent
losses that may be caused by the same direction changes in precious metal
prices.
  \\*

\end{eabstract}

\ekeywords{Realized Garch; Precious metals; Fluctuation characteristics; Dynamic correlation}


